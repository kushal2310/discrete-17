\let\negmedspace\undefined
\let\negthickspace\undefined
\documentclass[journal,12pt,twocolumn]{IEEEtran}
\usepackage{cite}
\usepackage{amsmath,amssymb,amsfonts,amsthm}
\usepackage{algorithmic}
\usepackage{graphicx}
\usepackage{textcomp}
\usepackage{xcolor}
\usepackage{txfonts}
\usepackage{listings}
\usepackage{enumitem}
\usepackage{mathtools}
\usepackage{gensymb}
\usepackage{comment}
\usepackage[breaklinks=true]{hyperref}
\usepackage{tkz-euclide} 
\usepackage{listings}
\usepackage{gvv}                                        
\def\inputGnumericTable{}                                 
\usepackage[latin1]{inputenc}                                
\usepackage{color}                                            
\usepackage{array}                                            
\usepackage{longtable}                                       
\usepackage{calc}                                             
\usepackage{multirow}                                         
\usepackage{hhline}                                           
\usepackage{ifthen}                                           
\usepackage{lscape}

\newtheorem{theorem}{Theorem}[section]
\newtheorem{problem}{Problem}
\newtheorem{proposition}{Proposition}[section]
\newtheorem{lemma}{Lemma}[section]
\newtheorem{corollary}[theorem]{Corollary}
\newtheorem{example}{Example}[section]
\newtheorem{definition}[problem]{Definition}
\newcommand{\BEQA}{\begin{eqnarray}}
\newcommand{\EEQA}{\end{eqnarray}}
\newcommand{\define}{\stackrel{\triangle}{=}}
\theoremstyle{remark}
\newtheorem{rem}{Remark}

\usepackage{graphicx}
\graphicspath{ {./Downloads/} }
\begin{document}

\bibliographystyle{IEEEtran}
\vspace{3cm}

\title{ASSIGNMENT 1}
\author{EE22BTECH11060 - TEJAVATH KUSHAL$^{*}$% <-this % stops a space
}
\maketitle
\newpage
\bigskip

\renewcommand{\thefigure}{\theenumi}
\renewcommand{\thetable}{\theenumi}


\maketitle
\subsection*{QUESTION 17:}
A man starts repaying a loan as first instalment of Rs. 100. If he increases the
instalment by Rs 5 every month, what amount he will pay in the 30th instalment?

\subsection*{ANSWER:}

Instalment paid by the man : 100

Instalment increased by the man per month: 5\\

The nth term of an arithmetic progression can be found using the formula:

 \begin{align}a_n= a+(n)d\label{eq:ap}\
 \end{align}
Here, \eqref{eq:ap}\\
\begin{center}
    \begin{table}[h]
    \begin{center}
\begin{tabular}{ | m{3cm} | m{4cm}| } 
  \hline
  a & first term  \\ 
  \hline
  n & number term \\ 
  \hline
  d & common difference   \\ 
  \hline
\end{tabular}
\end{center}
   \end{table}
\end{center}
In this case, the first term (a) is 100 and the common difference (d) is 5.\\

The amount paid in the 30th instalment:\\
\begin{align*}a_n= a+(n)d\label{eq:ap}\
 \end{align*}

Here, n=29\\
${a_2}_9=100+(29*5)$\\
${a_2}_9=100+145$ \\
${a_2}_9=245$\\

Therefore, the man will pay Rs. 245 in the 30th instalment.

\end{document}
